\documentclass[11pt,a4paper]{article}

\usepackage{epsfig}
\usepackage{multicol}

\usepackage[utf8]{inputenc}
\usepackage[brazil]{babel}
\usepackage{fancyheadings}
\usepackage{amsmath}
\usepackage{enumerate}
\DeclareGraphicsExtensions{.png,.pdf}
\usepackage{amsmath, amsfonts, amssymb}
\usepackage{graphicx}
\usepackage{multicol}
\usepackage[utf8]{inputenc}

% As margens
\setlength{\textheight}{24.0cm}
\setlength{\textwidth}{17.5cm}
\setlength{\oddsidemargin}{2.0cm} % Margens reais desejadas
\setlength{\evensidemargin}{2.0cm} % 2+17.5+1.5=21cm (largura A4)
\setlength{\topmargin}{1.5cm} % 1.5+1.6+1.0+24.0+1.6=29.7cm
\setlength{\headheight}{1.6cm} % (altura A4)
\setlength{\headsep}{1.0cm}
\setlength{\columnsep}{1.5cm} % Coluna = 8cm ((17.5-1.5)/2)
\addtolength{\oddsidemargin}{-1in}
\addtolength{\evensidemargin}{-1in}
\addtolength{\topmargin}{-1in}
\setlength{\footskip}{0.0cm}
\usepackage{tasks}
\usepackage{enumitem} 

% Novos comandos
\newcommand{\limite}{\displaystyle\lim}
\newcommand{\integral}{\displaystyle\int}
\newcommand{\somatorio}{\displaystyle\sum}

\pagestyle{fancy}


\usepackage{lipsum}

\lhead{
\includegraphics[width=1cm]{brasao.png}
}

\rhead{ 
\sc\textbf{U}niversidade \textbf{F}ederal do \textbf{C}eará\\
Campus Quixadá\\ Monitoria de Cálculo II, III e EDO}

\cfoot{}

\begin{document}

\begin{center}
		\Large Lista 3 - Funções de Várias variáveis e Curvas de Nível.
	\end{center}

\begin{enumerate}

\item Seja $A$ um conjunto definido por: $A = \{(x,y) \in \mathbb{R}^2 \,|\, y = 2x + 1\}$. Faça uma representação geométrica deste conjunto.

\item Seja $B$ um conjunto definido por: $B = \{(x,y) \in \mathbb{R}^2 \,|\, y \geq 2x + 1\}$. Faça uma representação geométrica deste conjunto.

\item Seja $C$ um conjunto definido por: $C = \{(x,y) \in \mathbb{R}^2 \,|\, x^2 + y^2 \leq 4 \}$. Faça uma representação geométrica deste conjunto.

\item Seja $D$ um conjunto definido por: $D = \{(x,y) \in \mathbb{R}^2 \,|\, x \geq 3 \}$. Faça uma representação geométrica deste conjunto.

\item Seja $E$ um conjunto definido por: $E = \{(x,y) \in \mathbb{R}^2 \,|\, y = x - 2 \}$. Faça uma representação geométrica deste conjunto.

\item Seja $F$ um conjunto definido por: $F = \{(x,y) \in \mathbb{R}^2 \,|\, y \geq x - 2 \}$. Faça uma representação geométrica deste conjunto.

\item Seja $G$ um conjunto definido por: $G = \{(x,y) \in \mathbb{R}^2 \,|\, y < x - 2 \}$. Faça uma representação geométrica deste conjunto.

\item Seja $H$ um conjunto definido por: $H = \{(x,y) \in \mathbb{R}^2 \,|\, x \geq 3 \}$. Faça uma representação geométrica deste conjunto.








\item Seja $I$ um conjunto definido por: $I = \{(x,y) \in \mathbb{R}^2 \,|\, y \geq x \}$. Faça uma representação geométrica deste conjunto.

\item Seja $J$ um conjunto definido por: $J = \{(x,y) \in \mathbb{R}^2 \,|\, x^2 + y^2 \leq 25 \}$. Faça uma representação geométrica deste conjunto.

\item Seja $K$ um conjunto definido por: $K = \{(x,y) \in \mathbb{R}^2 \,|\, x^2 + y^2 > 25 \}$. Faça uma representação geométrica deste conjunto.

\item Seja $L$ um conjunto definido por: $L = \{(x,y) \in \mathbb{R}^2 \,|\, (x - 2)^2 + y^2 < 1 \}$. Faça uma representação geométrica deste conjunto.

\item Seja $M$ um conjunto definido por: $M = \{(x,y) \in \mathbb{R}^2 \,|\, (x - 4)^2 + (y - 4)^2 \leq 1 \}$. Faça uma representação geométrica deste conjunto.

\item Obtenha os pontos que satisfazem simultaneamente as relações: 
	\begin{enumerate}[label=(\roman*)]
	\item $y \geq x + 2$
	\item $y \geq 2$
	\end{enumerate}
	
\item Obtenha os pontos que satisfazem simultaneamente as relações: 
	\begin{enumerate}[label=(\roman*)]
	\item $x + y \geq 2$
	\item $-x + y \geq 2$
	\end{enumerate}
	
\item Obtenha os pontos que satisfazem simultaneamente as relações: 
	\begin{enumerate}[label=(\roman*)]
	\item $x + y \leq 10$
	\item $x \leq 4$
	\end{enumerate}
	
	
	
	
	
	
	
	
	
\item Obtenha os pontos que satisfazem simultaneamente as relações: 
	\begin{enumerate}[label=(\roman*)]
	\item $|x| \leq 3$
	\item $|y| \leq 2$
	\end{enumerate}
	
\item Obtenha os pontos que satisfazem simultaneamente as relações: 
	\begin{enumerate}[label=(\roman*)]
	\item $x^2 + y^2 \leq 9$
	\item $x + y \geq 3$
	\end{enumerate}

\item Seja $A$ um conjunto definido por: $A = \{(x,y) \in \mathbb{R}^2 \,|\, y \geq x^2 \}$. Faça uma representação geométrica deste conjunto.

\item Seja $B$ um conjunto definido por: $B = \{(x,y) \in \mathbb{R}^2 \,|\, y \geq x^2 + 1 \}$. Faça uma representação geométrica deste conjunto.

\item Seja $C$ um conjunto definido por: $C = \{(x,y) \in \mathbb{R}^2 \,|\, y \leq 1 - x^2 \}$. Faça uma representação geométrica deste conjunto.

\item Seja $D$ um conjunto definido por: $D = \{(x,y) \in \mathbb{R}^2 \,|\, y \geq \frac{1}{x}\ \textrm{e } x > 0 \}$. Faça uma representação geométrica deste conjunto.

\item Seja $E$ um conjunto definido por: $E = \{(x,y) \in \mathbb{R}^2 \,|\, y \geq \frac{2}{x}\ \textrm{e } x > 0 \}$. Faça uma representação geométrica deste conjunto.

\item Esboce o gráfico da relação: $A = \{(x,y,z) \in \mathbb{R}^3 \,|\, x = 3 \}$.










\item Esboce o gráfico da relação: $B = \{(x,y,z) \in \mathbb{R}^3 \,|\, y = 2 \}$.
\item Esboce o gráfico da relação: $C = \{(x,y,z) \in \mathbb{R}^3 \,|\, x = 2 \}$.
\item Esboce o gráfico da relação: $D = \{(x,y,z) \in \mathbb{R}^3 \,|\, x = 0 \}$.
\item Esboce o gráfico da relação: $E = \{(x,y,z) \in \mathbb{R}^3 \,|\, y = 0 \}$.

\item Esboce o gráfico do plano: $x + y + z = 2$.
\item Esboce o gráfico do plano: $2x + 3y + 4z -12 = 0$.
\item Esboce o gráfico do plano: $3x + 4y - z - 12 = 0$.
\item Esboce o gráfico do plano: $x - y + z - 1 = 0$.





\item Esboce o gráfico do plano: $x - y = 0$.
\item Esboce o gráfico do plano: $x + y = 2$.
\item Calcule a distância entre os pontos A e B: $A(1,2,1)\ \textrm{e } B(4,2,3)$.
\item Calcule a distância entre os pontos A e B: $A(2,1,3)\ \textrm{e } B(0,0,0)$.
\item Calcule a distância entre os pontos A e B: $A(-1,2,-1)\ \textrm{e } B(0,1,-3)$.
\item Calcule a distância entre os pontos A e B: $A(1,4,2)\ \textrm{e } B(1,4,2)$.
\item Calcule a distância entre os pontos A e B: $A(1,0,2,3)\ \textrm{e } B(2,3,1,1)$.
\item Calcule a distância entre os pontos A e B: $A(-1,2,0,0,3)\ \textrm{e } B(0,0,-2,-1,5)$.










\item Calcule a distância entre os pontos A e B: $A(1,2,3,4)\ \textrm{e } B(1,2,3,4)$.

\item Dado o conjunto $A = \{(x,y) \in \mathbb{R}^2 \,|\, x \leq 2\}$, assinale os pontos interiores:
	\begin{enumerate}
		\item $P(1,4)$
		\item $P(9/2,5)$
		\item $P(2,6)$
		\item $P(3,9)$
		\item $P(2,2)$
		\item $P(1,-9)$
	\end{enumerate}	
	
\item Dado o conjunto $A = \{(x,y) \in \mathbb{R}^2 \,|\, x + y \leq 2, x \geq 0\ \textrm{e } y \geq 0 \}$, assinale os pontos interiores:
	\begin{enumerate}
		\item $P(1,0)$
		\item $P(0,1)$
		\item $P(1,1)$
		\item $P(1/2,1/2)$
		\item $P(1/2,1/4)$
		\item $P(3,0)$
	\end{enumerate}		

\item O conjunto $A = \{(x,y) \in \mathbb{R}^2 \,|\, y > 3 \}$ é aberto? Justifique.

\item O conjunto $B = \{(x,y) \in \mathbb{R}^2 \,|\, |x| < 3 \}$ é aberto? Justifique.

\item O conjunto $C = \{(x,y) \in \mathbb{R}^2 \,|\, y \geq x \}$ é aberto? Justifique.

\item O conjunto $D = \{(x,y) \in \mathbb{R}^2 \,|\, y = x + 1 \}$ é aberto? Justifique.

\item O conjunto $D = \{(x,y,z) \in \mathbb{R}^3 \,|\, 0 < x < 1, 0 < y < 1, 0 < z < 1 \}$ é aberto? Justifique.











\item O conjunto $E = \{(x,y,z) \in \mathbb{R}^3 \,|\, 0 \leq x \leq 1, 0 \leq y \leq 1, 0 \leq z \leq 1 \}$ é aberto? Justifique.

\item Identifique os pontos de fronteira do conjunto: $A = \{(x,y) \in \mathbb{R}^2 \,|\, y \geq 5 \}$.

\item Identifique os pontos de fronteira do conjunto: $B = \{(x,y) \in \mathbb{R}^2 \,|\, -1 \leq x \leq 2 \}$.

\item Identifique os pontos de fronteira do conjunto: $C = \{(x,y) \in \mathbb{R}^2 \,|\, 0 < y \leq 2 \}$.

\item Identifique os pontos de fronteira do conjunto: $D = \{(x,y) \in \mathbb{R}^2 \,|\, y \geq x^2 \}$.

\item Identifique os pontos de fronteira do conjunto: $E = \{(x,y) \in \mathbb{R}^2 \,|\, y \leq x^2 - 1 \}$.

\item Identifique os pontos de fronteira do conjunto: $F = \{(x,y) \in \mathbb{R}^2 \,|\, x^2 + y^2 \leq 9 \}$.

\item Identifique os pontos de fronteira do conjunto: $G = \{(x,y) \in \mathbb{R}^2 \,|\, y \geq x \}$.













\item Identifique os pontos de fronteira do conjunto: $H = \{(x,y) \in \mathbb{R}^2 \,|\, y \geq 1/x \}$.

\item Defina o domínio da função $z = f(x,y) = \displaystyle\frac{x + y}{x - y}$ matematicamente e geometricamente.

\item Defina o domínio da função $z = f(x,y) = \sqrt[4]{1 - x^2 - y^2}$ matematicamente e geometricamente.

\item Defina o domínio da função $z = f(x,y) = \displaystyle\frac{2x + 3y}{\sqrt{1-x-y}}$ matematicamente e geometricamente.

\item Defina o domínio da função $z = f(x,y) = \displaystyle\frac{2x + 3y}{\sqrt{1-x-y}}$ matematicamente e geometricamente.

\item Defina o domínio da função $z = f(x,y) = \displaystyle\frac{\sqrt{x-y}}{\sqrt{1-x^2-y^2}}$ matematicamente e geometricamente.

\item Defina o domínio da função $z = f(x,y) = \sqrt{36 - x^2 - y^2 - z^2}$ matematicamente e geometricamente.

\item Como é definida a função $f(x,y) = 2x^2y^3 + 3x^2y + 1$ enquanto a sua natureza?














\item Como é definida a função $g(x,y) = 3xy + 4y - 1$ enquanto a sua natureza?

\item Como é definida a função $f(x,y) = ax + by$, enquanto a sua natureza, onde $(a,b) \in \mathbb{R} $?

\item A função $f(x,y) = \displaystyle\frac{x^2}{x^2 + y^2}$ é limitada? Justifique.

\item A função $f(x,y) = x^2 + y^2$ é homogênea? Justifique.

\item A função $f(x,y) = \displaystyle\frac{xye^{x/y}}{x^2 + y^2}$ é homogênea? Justifique.

\item Encontre o conjunto de nível da função $f(x,y) = x - y$.

\item Encontre o conjunto de nível da função $f(x,y) = x^2 + y^2$.

\item Encontre o conjunto de nível da função $f(x,y) = y^2 - x^2$.












\item Encontre o conjunto de nível da função $f(x,y,z) = x^2 + y^2 + z^2$.

\item Defina o domínio da função $f(x,y) = \displaystyle\frac{2xy^2}{x^2 + y^4}$. Em seguida encontre as Curvas de Nível da função.

\item Esboce as curvas de nível da função $f(x,y) = 3x + 4y$ nos níveis $c = 12$ e $c = 24$.

\item Esboce as curvas de nível da função $f(x,y) = x - y$ nos níveis $c = 0$, $c = 1$ e $c = -1$.

\item Esboce as curvas de nível da função $f(x,y) = \displaystyle\frac{1}{x^2 + y^2}$ nos níveis $c = 1$, e $c = 4$.

\item Esboce as curvas de nível da função $f(x,y) = y - x^2 + 4$ nos níveis $c = 0$, e $c = 5$.

\item Esboce as curvas de nível da função $f(x,y) = y - x^3$ nos níveis $c = 0$, e $c = 1$.

\item Esboce as curvas de nível da função $g(x,y) = \sqrt{9 - x^2 - y^2}$ nos níveis $c = 0, c = 1, c = 2$, e $c = 3$.

\end{enumerate}

\end{document}