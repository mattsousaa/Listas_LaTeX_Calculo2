\documentclass[11pt,a4paper]{article}

\usepackage{epsfig}
\usepackage{multicol}

\usepackage[utf8]{inputenc}
\usepackage[brazil]{babel}
\usepackage{fancyheadings}
\usepackage{amsmath}
\usepackage{enumerate}
\DeclareGraphicsExtensions{.png,.pdf}
\usepackage{amsmath, amsfonts, amssymb}
\usepackage{graphicx}
\usepackage{multicol}
\usepackage[utf8]{inputenc}

% As margens
\setlength{\textheight}{24.0cm}
\setlength{\textwidth}{17.5cm}
\setlength{\oddsidemargin}{2.0cm} % Margens reais desejadas
\setlength{\evensidemargin}{2.0cm} % 2+17.5+1.5=21cm (largura A4)
\setlength{\topmargin}{1.5cm} % 1.5+1.6+1.0+24.0+1.6=29.7cm
\setlength{\headheight}{1.6cm} % (altura A4)
\setlength{\headsep}{1.0cm}
\setlength{\columnsep}{1.5cm} % Coluna = 8cm ((17.5-1.5)/2)
\addtolength{\oddsidemargin}{-1in}
\addtolength{\evensidemargin}{-1in}
\addtolength{\topmargin}{-1in}
\setlength{\footskip}{0.0cm}
\usepackage{tasks}

% Novos comandos
\newcommand{\limite}{\displaystyle\lim}
\newcommand{\integral}{\displaystyle\int}
\newcommand{\somatorio}{\displaystyle\sum}

\pagestyle{fancy}


\usepackage{lipsum}

\lhead{
\includegraphics[width=1cm]{brasao.png}
}

\rhead{ 
\sc\textbf{U}niversidade \textbf{F}ederal do \textbf{C}eará\\
Campus Quixadá\\ Monitoria de Cálculo II e III}

\cfoot{}

\begin{document}

	\begin{center}
		\Large Lista 8 - Cálculo II
	\end{center}
	
	Determine as derivadas parciais	
	
	\begin{enumerate}

	
	\begin{enumerate}
		\item $f(x,y) = 5x^4y^2 + xy^3 + 4$
		\item $\displaystyle\frac{x^3 + y^2}{x^2 + y^2}$
		\item $z = x^2 \ln (1 + x^2 + y^2)$
		\item $f(x,y) = (4xy - 3y^3)^3 + 5x^2y$
		\item $g(x,y) = x^y$
		\item $f(x,y) = \sqrt[3]{x^3 + y^2 + 3}$
		\item $\cos xy$
		\item $f(x,y) = e^{-x^2 - y^2}$
		
		
		
		
		
		
		
		
		
		
		
		
		
		
		\item $z = xye^{xy}$
		\item $z = \arctan x/y$
		\item $(x^2 + y^2) \ln(x^2 + y^2)$
		\item $z = \displaystyle\frac{x\sin y}{\cos (x^2 + y^2)}$
		
	\end{enumerate}
	
	\item Considere a função $z = \displaystyle\frac{xy^2}{x^2 + y^2}$. Verifique que $x\dfrac{\partial z}{\partial x} + y \dfrac{\partial z}{\partial y}$
	
	 Seja $\phi : \mathbb{R} \to \mathbb{R}$ uma função de uma variável real, diferenciável e tal que $\phi '(1) = 4$. 
	
	Seja $g(x,y) = \phi \left(\displaystyle\frac{x}{y}\right)$. Calcule
	
	\begin{enumerate}
		\item $\dfrac{\partial g}{\partial x}(1,1)$
		\item $\dfrac{\partial g}{\partial y}(1,1)$
	\end{enumerate}
	
	\item Seja $g(x,y) = \phi \left(\displaystyle\frac{x}{y}\right)$. Verifique que:
	
	$$x\dfrac{\partial g}{\partial x}(x,y) + y\dfrac{\partial g}{\partial y}(x,y) = 0$$	
	
	
	
	
	
	
	
	
	
	
	
	
	
	
	
	
	
	
	
	
	
	\item A função $p = p(V,T)$ é dada implicitamente pela equação $pV = nRT$, onde $n$ e $R$ são constantes não-nulas. Calcule $\dfrac{\partial p}{\partial V}$ e $\dfrac{\partial p}{\partial T}$. 
	
	\item Seja $z = e^y \phi (x-y)$, onde $\phi$ é uma função diferenciável de uma variável real. Mostre que
	$$\dfrac{\partial z}{\partial x} + \dfrac{\partial z}{\partial y} = z$$
	
	\item Seja $\phi : \mathbb{R} \to \mathbb{R}$ uma função diferenciável de uma variável real e seja $f(x,y) = (x^2 + y^2)\phi \left(\displaystyle\frac{x}{y}\right)$. Mostre que
	%$$x\dfrac{\partial f}{\partial x} + y \dfrac{\partial f}{\partial y} = 2f$$	
	
	\item Sejam $z = e^{x^2 + y^2}$, $x = \rho \cos \theta$ e $y = \rho \sin \theta$. Verifique que 
	$$\dfrac{\partial z}{\partial p} = e^{x^2 + y^2} (2x \cos \theta + 2y \sin \theta)$$
	Conclua que
	$$\dfrac{\partial z}{\partial p} = \dfrac{\partial z}{\partial x}\cos \theta + \dfrac{\partial z}{\partial y}\sin \theta $$	
	 
	\item Suponha que a função $z = z(x,y)$ admita derivadas parciais em todos os pontos de seu domínio e que seja dada implicitamente pela equação $xyz + z^3 = x$. Expresse $\dfrac{\partial z}{\partial x}$ e $\dfrac{\partial z}{\partial y}$ em termos de $x,y,z$. 
	
	\item Seja $f(x,y) = \integral_0^{x^2 + y^2} e^{-t^2} \, dt$. Calcule $\dfrac{\partial f}{\partial x}(x,y)$ e $\dfrac{\partial f}{\partial y}(x,y)$.
	
	\item Seja $f(x,y) = \integral_{x^2}^{  y^2} e^{-t^2} \, dt$. Calcule $\dfrac{\partial f}{\partial x}(x,y)$ e $\dfrac{\partial f}{\partial y}(x,y)$.
	
	\item Considere a função $z = f(x,y)$ e seja $(x_0, y_0) \in Df$. Como você definiria plano tangente ao gráfico de $f$ no ponto $(x_0, y_0)$? Admitindo que $f$ admita derivadas parciais em $(x_0, y_0)$, escreva a equação de um plano que você acha que seja um "forte" candidato a plano tangente ao gráfico de $f$ no ponto $(x_0, y_0, f(x_0,y_0))$.
	
	
	
	
	
	
	
	
	
	
	
	
	
	
	
	
	
	
	
	\item Seja $f(x,y) = x^3y^2 - 6xy + \phi (y)$. Determine uma função $\phi$ de modo que
	$$\dfrac{\partial f}{\partial y} = 2x^3y - 6x + \displaystyle\frac{y}{y^2 + 1}$$.
	
	\item Seja	
	
	$$f(x,y) = 
		\begin{cases}
			e^{\displaystyle\frac{1}{x^2 + y^2 - 1}}\, \textrm{se }x^2 + y^2 < 1 \\
			0\,\, \quad \quad \quad \quad \quad \textrm{se }x^2 + y^2 \geq 1 \\
		\end{cases}
	%$$
	\begin{enumerate}
		\item Esboce o gráfico da função.
		\item Determine $\dfrac{\partial f}{\partial x}$ e $\dfrac{\partial f}{\partial y}$.
	\end{enumerate}
	
	Seja $f : \mathbb{R} \to \mathbb{R}$ contínua com $f(3) = 4$. Seja 
	$$g(x,y,z) = \integral_0^{x + y^2 + z^4} f(t) \, dt$$.
	
	\begin{enumerate}
		\item $\dfrac{\partial g}{\partial x}(1,1,1)$
		\item $\dfrac{\partial g}{\partial y}(1,1,1)$
		\item $\dfrac{\partial g}{\partial z}(1,1,1)$
	\end{enumerate}
	
	\item O plano x = 1 apresenta intersecção com o paraboloide $z = x^2 + y^2$ em uma parábola. Encontre o coeficiente angular da tangente à parábola em $(1,2,5)$.
	
	\item Se resistores elétricos de $R1$, $R2$ e $R3$ ohms são conectados em paralelo para formar um resistor de $R$ ohms, o valor de $R$ pode ser encontrado a partir da equação
	$$\frac{1}{R} = \displaystyle\frac{1}{R1} + \displaystyle\frac{1}{R2} + \displaystyle\frac{1}{R3}$$
	Encontre o valor de $\dfrac{\partial R}{\partial R2}$ quando $R_1 = 30$, $R_2 = 45$ e $R_3 = 90$ ohms.
	
	
	
	
	
	
	
	
	
	
	
	
	
	
	
	
	
	
	
	
	
	
	
	
	
	 
	
	 Se $f(x,y) = x \cos y + ye^x$, encontre as derivadas de segunda ordem a seguir:
	
	\begin{enumerate}
		\item $\dfrac{\partial^2 f}{\partial x^2}$
		\item $\dfrac{\partial^2 f}{\partial y^2}$
		\item $\dfrac{\partial }{\partial y}\left(\dfrac{\partial f}{\partial x}\right)$
		
		\item $\dfrac{\partial }{\partial x}\left(\dfrac{\partial f}{\partial y}\right)$
	\end{enumerate}	 
	
		\item Seja a função $w = xy + \displaystyle\frac{e^y}{y^2 + 1}$. Encontre $\dfrac{\partial }{\partial x}\left(\dfrac{\partial w}{\partial y}\right)$.
	
		 Seja a função $f(x,y,z) = 1 - 2xy^2z + x^2y$. Encontre as derivadas parciais a seguir:
		
		\begin{enumerate}
			\item $f_y$
			\item $f_{yx}$
			\item $f_{yxy}$
			
			
			
			
			
			
			
			
			
			
			
			
			
			
			
			
			
			
			
			
			
			
			\item $f_{yxyz}$
			
		\end{enumerate}
		
		
		 Encontre $\dfrac{\partial f}{\partial x}$ e $\dfrac{\partial f}{\partial y}$ nos itens a seguir.
		
		\begin{enumerate}
			\item $f(x,y) = 2x^2 - 3y - 4$.
			\item $f(x,y) = x^2 - xy + y^2$.
			\item $f(x,y) = (x^2 - 1)(y + 2)$.
			\item $f(x,y) = 5xy - 7x^2 - y^2 + 3x - 6y + 2$.
			\item $f(x,y) = (xy - 1)^2$.
			\item $f(x,y) = (x^3 + (y/2))^{2/3}$.
			\item $f(x,y) = \frac{1}{x + y} $.
			
			
			
			
			
			
			
			
			
			
			
			
			
			
			
			
			
			
			\item $f(x,y) = \frac{x}{x^2 + y^2} $.
			\item $f(x,y) = \frac{x+y}{xy - 1} $.
		\item $f(x,y) = e^{x + y + 1}$.
			\item $f(x,y) = e^{-x} \sin (x+y) $.
		\item $f(x,y) = \ln (x + y)$.
			\item $f(x,y) = \arctan \frac{y}{x}$.
			\item $f(x,y) = e^{xy} \ln (x + y)$.
			\item $f(x,y) = \sin^2 (x - 3y)$.
			
			
			
			
			
			
			
			
			
			
			
			
			
			
			\item $f(x,y) = \cos^2 (3x - y^2)$.
			\item $f(x,y) = \log_y x$.
			\item $f(x,y) = x + y + xy$.
			\item $f(x,y) = \sin (xy + 1/x)$.
			\item $f(x,y) = \cos y + y\sin x$.
			\item $f(x,y) = x^2 \tan (xy)$.
			\item $f(x,y) = x\sin (x^2y)$.
			\item $f(x,y) = \ln (2x + 3y)$.
			
			
			
			
			
			
			
			
			
			
			
		\end{enumerate}
		
		 A equação de Laplace
		tridimensional 
		$$\dfrac{\partial^2 f}{\partial x^2} +\dfrac{\partial^2 f}{\partial y^2} + \dfrac{\partial^2 f}{\partial z^2} = 0$$
		é satisfeita pelas distribuições de temperatura no estado estacionário $T = f(x,y,z)$ no espaço, 				pelos potenciais gravitacionais e pelos potenciais eletrostáticos. A equação de Laplace 						bidimensional
		$$\dfrac{\partial^2 f}{\partial x^2} +\dfrac{\partial^2 f}{\partial y^2} = 0$$
		obtida eliminando-se o termo $\dfrac{\partial^2 f}{\partial z^2}$ da equação anterior, descreve
	    potenciais e distribuições de temperatura no estado estacionário no
		plano.
		
		Mostre que cada função a seguir satisfaz uma equação de Laplace.
		
		\begin{enumerate}
			\item $f(x,y,z) = x^2 + y^2 - 2z^2$
		\item $f(x,y,z) = 2z^3 - 3(x^2 + y^2)z$
			\item $f(x,y) = e^{-2y}\cos 2x$
			\item $f(x,y) = \ln \sqrt{x^2 + y^2}$
		\item $f(x,y) = 3x + 2y - 4$
			\item $f(x,y) = \arctan \displaystyle\frac{x}{y} $
			\item $f(x,y,z) = (x^2 + y^2 + z^2)^{-1/2}$
			\item $f(x,y,z) = e^{3x + 4y} \cos 5z$
			
			
			
			
			
			
			
			
			
			
			
			
			
			
			
			
			
			
			
			
			
			
			
		\end{enumerate}
		
		 Se ficarmos em uma praia e tirarmos uma fotografia das ondas, essa foto mostrará um padrão regular de picos
e depressões em dado instante. Veremos movimento vertical periódico no espaço em relação à distância. Se ficarmos na água, poderemos sentir a subida e descida da água com o passar das ondas. Veremos movimento periódico vertical no tempo. Na física, essa
bela simetria é expressa pela equação de onda unidimensional
	$$\dfrac{\partial^2 w}{\partial t^2} = c^2\dfrac{\partial^2 w}{\partial x^2} $$
	em que $w$ é a altura da onda, $x$ é a distância, $t$ é o tempo e $c$ é a velocidade com a qual as ondas se propagam. Em nosso exemplo, x é a distância ao longo da superfície do mar,mas em outras aplicações x pode ser a distância ao longo de uma corda vibrando, a distância no ar (ondas sonoras) ou a distância no
espaço (ondas luminosas). O número c varia de acordo com o meio e o tipo de onda.
Mostre que as funções abaixo são todas soluções da equação de onda.

	\begin{enumerate}
		\item $w = \sin (x + ct)$
		\item $w = \cos (2x + 2ct)$
		\item $w = \sin (x + ct) + \cos (2x + 2ct)$
		\item $w = \ln (2x + 2ct)$
		\item $w = \tan (2x - 2ct)$
		\item $w = 5\cos(3x + 3ct) + e^{x + ct}$
	\end{enumerate}
	
	\textbf{DESAFIO}
	
	\item Na física o estudo de oscilações é muito importante e está muito presente dentro do nosso dia a dia. Um dos problemas mais clássicos que temos é a descrição vibratória de fenômenos oscilatórios, como a vibração de uma das cordas de um violão, por exemplo. Imagine uma corda presa em suas extremidades. Essa corda quando está relaxada possui comprimento L. Alguém a vibra de forma que seu movimento é de forma vibratória e estacionária. Mostre que com as informações abaixo você pode demonstrar a natureza matemática de uma corda em vibração. No final de tudo, esclareça suas conclusões e demonstre o significado das contas. Cálculo com número não é significado. Interprete. 
	
	Uma corda vibra de acordo com uma função $u = u(x,t)$, onde $t$ é o tempo e $x$ é a posição que a onda se encontra do eixo x. Imagine uma corda vibrando ao longo de um plano cartesiano $xy$, onde o eixo y é equivalente a $u$ e o eixo x é equivalente ao tempo. Temos condições de fronteira que modelam o nosso problema, e podem ser definidas como:
	$$u(x,y) = 
		\begin{cases}
			u_{tt} = c^2u_{xx} \\
			u(0,t) = 0 \\
			u(L,t) = 0 \\
			u(x,0) = f(x) \\
			u_t(x,0) = g(x)
		\end{cases}
	$$
	
	$u_{tt} = c^2u_{xx}$ é a equação que descreve movimentos oscilatórios em geral. O método de resolução é por separação de variáveis. Imagine que $u(x,t)$ pode ser decomposta em outras duas funções:
	$$u(x,t) = \phi (t) \psi (t) $$
	A ideia é fazer por partes. Na equação acima você pode usar:
	$$u_{tt} = \phi (x) \psi ''(t) $$
	$$u_{xx} = \psi (t) \phi ''(x) $$
	
	Utilize cada condição de fronteira nas 2 equações acima. Mostre que através disso, podemos descrever $u(x,t)$ como:
	
$$u(x,t) = \displaystyle\frac{8a}{\pi ^2} \somatorio_{n=1}^{\infty} \displaystyle\frac{(-1)^n}{(2n + 1)^2} \cos \displaystyle\left(\frac{(2n + 1)\pi t}{2}\right) \sin \displaystyle\left(\frac{(2n + 1)\pi x}{2}\right)$$
	
	\item O volume $V$ de um cilindro reto é dado pela fórmula $V = \pi r^2 h$, onde $r$ é o raio e h é a altura. Determine uma fórmula para a taxa de variação instantânea de $V$ em relação a $r$ se $r$ variar e $h$ permanecer constante. 
	
	 \item Determine uma relação da área lateral de um cone quando temos a altura desse cone variando no tempo. 
	
	\end{enumerate}
	

\end{document}