\documentclass[11pt,a4paper]{article}

\usepackage{epsfig}
\usepackage{multicol}

\usepackage[utf8]{inputenc}
\usepackage[brazil]{babel}
\usepackage{fancyheadings}
\usepackage{amsmath}
\usepackage{enumerate}
\DeclareGraphicsExtensions{.png,.pdf}
\usepackage{amsmath, amsfonts, amssymb}
\usepackage{graphicx}
\usepackage{multicol}
\usepackage[utf8]{inputenc}

% As margens
\setlength{\textheight}{24.0cm}
\setlength{\textwidth}{17.5cm}
\setlength{\oddsidemargin}{2.0cm} % Margens reais desejadas
\setlength{\evensidemargin}{2.0cm} % 2+17.5+1.5=21cm (largura A4)
\setlength{\topmargin}{1.5cm} % 1.5+1.6+1.0+24.0+1.6=29.7cm
\setlength{\headheight}{1.6cm} % (altura A4)
\setlength{\headsep}{1.0cm}
\setlength{\columnsep}{1.5cm} % Coluna = 8cm ((17.5-1.5)/2)
\addtolength{\oddsidemargin}{-1in}
\addtolength{\evensidemargin}{-1in}
\addtolength{\topmargin}{-1in}
\setlength{\footskip}{0.0cm}
\usepackage{tasks}

% Novos comandos
\newcommand{\limite}{\displaystyle\lim}
\newcommand{\integral}{\displaystyle\int}
\newcommand{\somatorio}{\displaystyle\sum}

\pagestyle{fancy}


\usepackage{lipsum}

\lhead{
\includegraphics[width=1cm]{brasao.png}
}

\rhead{ 
\sc\textbf{U}niversidade \textbf{F}ederal do \textbf{C}eará\\
Campus Quixadá\\ Monitoria de Cálculo II, III e EDO}

\cfoot{}

\begin{document}

	\begin{center}
		\Large Lista 7 - Regra da Cadeia
	\end{center}
	
	\begin{enumerate}
	
	\item Use a Regra da Cadeia para achar $\dfrac{\partial z}{\partial s}$ e $\dfrac{\partial z}{\partial t}$.
	
	\begin{enumerate}
		\item $z = x^2y^3$, \quad $x = s \cos t$, $y = s \sin t$
		\item $z = \arcsin (x - y)$, \quad $x = s^2 + t^2$, $y = 1 - 2st$
		\item $z = \sin \theta \cos \phi$, \quad $\theta = st^2$, $\phi = s^2t$
		\item $z = e^{x+2y}$, \quad $x = s/t$, $y = t/s$
		\item $z = \tan (u/v)$, \quad $u = 2s + 3t$, $v = 3s - 2t$
	\end{enumerate}
	
	\item Utilize um diagrama de árvore para escrever a Regra da Cadeia para o caso dado. Suponha que todas as funções sejam diferenciáveis.
	\begin{enumerate}
		\item $u = f(x,y)$, onde $x = x(r,s,t)$, $y = y(r,s,t)$
		\item $R = f(x,y,z,t)$, onde $x = x(u,v,w)$, $y = y(u,v,w)$, $z = z(u,v,w)$, $t = t(u,v,w)$
		\item $w = f(r,s,t)$, onde $r = r(x,y)$, $s = s(x,y)$, $t = t(x,y)$
		\item $t = f(u,v,w)$, onde $u = u(p,q,r,s)$, $v = v(p,q,r,s)$, $w = w(p,q,r,s)$
		
	\end{enumerate}
	
	\item Utilize a Regra da Cadeia para determinar as derivadas parciais indicadas.
	\begin{enumerate}
		\item $z = x^2 + xy^3$, onde $x= uv^2 + w^3$ e $y = u + ve^w$; 
		
		Procure por $\dfrac{\partial z}{\partial u}$, $\dfrac{\partial z}{\partial v}$, $\dfrac{\partial z}{\partial w}$ quando $u = 2$, $v = 1$, $w = 0$.
		\item $u = \sqrt{r^2 + s^2}$, onde $r = y + x \cos t$ e $s = x + y \sin t$; 
		
		Procure por $\dfrac{\partial u}{\partial x}$, $\dfrac{\partial u}{\partial y}$, $\dfrac{\partial u}{\partial t}$ quando $x = 1$, $y = 2$, $t = 0$.
		\item $w = xy + yz + zx$, onde $x = r \cos \theta $, $y = r \sin \theta$ e $z = r \theta $; 
		
		Procure por $\dfrac{\partial w}{\partial r}$, $\dfrac{\partial w}{\partial \theta}$, quando $r = 2$, $\theta = \pi / 2$.
		\item $P = \sqrt{u^2 + v^2 + w^2}$, onde $u = xe^y$, $v = ye^x$ e $w = e^{xy}$; 
		
		Procure por $\dfrac{\partial P}{\partial x}$, $\dfrac{\partial P}{\partial y}$, $x = 0$, $y = 2$.
		\item $N = \displaystyle\frac{p + q}{p + r}$, onde $p = u + vw$, $q = v + uw$ e $r = w + uv$;
		
		 Procure por $\dfrac{\partial N}{\partial u}$, $\dfrac{\partial N}{\partial v}$, $\dfrac{\partial N}{\partial w}$ quando $u = 2$, $v = 3$, $w = 4$.
		
		\item $u = xe^{ty}$, onde $x = \alpha^2 \beta$, $y = \beta^2\gamma$ e $t = \gamma^2 \alpha$;
		
		 Procure por $\dfrac{\partial u}{\partial \alpha}$, $\dfrac{\partial u}{\partial \beta}$, $\dfrac{\partial u}{\partial \gamma}$ quando $\alpha = -1$, $\beta = 2$, $\gamma = 1$.
	\end{enumerate}
	
	\item Se $u = f(x,y)$, onde $x = e^s\cos t$ e $y = e^s \sin t$, mostre que
	$$\dfrac{\partial^2 u}{\partial x^2} + \dfrac{\partial^2 u}{\partial y^2} = e^{-2s}\left[\dfrac{\partial^2 u}{\partial s^2} + \dfrac{\partial^2 u}{\partial t^2}\right]$$
	
	 \item Se $z = f(x,y)$, onde $x = r^2 + s^2$, $y = 2rs$, determine $\dfrac{\partial^2 z}{\partial r \partial s}$
	 
	 \item Se $z = f(x,y)$, onde $x = r \cos \theta$, e $y = r \sin \theta$, determine
	 \begin{enumerate}
	 	\item $\dfrac{\partial z}{\partial r}$
	 	\item $\dfrac{\partial z}{\partial \theta}$
	 	\item $\dfrac{\partial^2 z}{\partial r \partial \theta}$
	 \end{enumerate}
	 
	 \item Se $z = f(x,y)$, onde $x = r \cos \theta$, e $y = r \sin \theta$, mostre que
	 $$\dfrac{\partial^2 z}{\partial x^2}+ \dfrac{\partial^2 z}{\partial y^2} = \dfrac{\partial^2 z}{\partial r^2} + \displaystyle\frac{1}{r^2}\dfrac{\partial^2 z}{\partial \theta^2} + \displaystyle\frac{1}{r}\dfrac{\partial z}{\partial r}$$
	
	 \item Suponha que $z = f(x,y)$, onde $x = g(s,t)$ e $y = h(s,t)$.
	 \begin{enumerate}
	 	\item $\dfrac{\partial^2 z}{\partial t^2} = \dfrac{\partial^2 z}{\partial x^2} \left(\dfrac{\partial x}{\partial t}\right)^2 + 2 \dfrac{\partial^2 z}{\partial x \partial y} \dfrac{\partial x}{\partial t} \dfrac{\partial y}{\partial t} + \dfrac{\partial^2 z}{\partial y^2} \left(\dfrac{\partial y}{\partial t}\right)^2 + \dfrac{\partial z}{\partial x} \dfrac{\partial^2 x}{\partial t^2} + \dfrac{\partial z}{\partial y} \dfrac{\partial^2 y}{\partial t^2} $
	 	
	 	\item Determine uma fórmula semelhante para $\dfrac{\partial^2 z}{\partial s \partial t}$.
	 	
	 \end{enumerate}
	 
	 \item Se um som com frequência $f_s$ for produzido por uma fonte se movendo ao longo de uma reta com velocidade $v_s$ e um observador estiver se movendo com velocidade $v_0$ ao longo da mesma reta a partir da direção oposta, em direção à fonte, então a frequência do som ouvido pelo observador é
	 $$f_0 = \left(\frac{c + v_0}{c - v_s}\right) f_s$$ 
	 
	 onde $c$ é a velocidade do som, cerca de 332 $m/s$. (Este é o efeito Doppler). Suponha que, em um dado momento, você esteja em um trem que se move a 34 $m/s$ e acelera a 1.2 $m/s^2$. Um trem se aproxima de você na direção oposta no outro trilho a 40 $m/s$, acelerando a 1.4 $m/s^2$, e toca seu apito, com frequência de 460 $Hz$. Neste instante, qual é a frequência aparente que ouve e quão rapidamente ela está variando?
	 
	 \item Nas partes (a)-(e), suponha que a equação $z = f(x,y)$ seja expressa na forma polar $z = g(r, \theta)$ fazendo a substituição $x = r\cos \theta$ e $y = r\sin \theta$.
	 
	\begin{enumerate}
		\item Considere $r$ e $\theta$ como funções de $x$ e $y$ e use derivação implícita para mostrar que
		$\dfrac{\partial r}{\partial x} = \cos \theta $ e $\dfrac{\partial \theta}{\partial x} = - \displaystyle\frac{\sin \theta}{r}$
		
		\item Considere $r$ e $\theta$ como funções de $x$ e $y$ e use derivação implícita para mostrar que
		$\dfrac{\partial r}{\partial y} = \sin \theta $ e $\dfrac{\partial \theta}{\partial y} = \displaystyle\frac{\cos \theta}{r}$
		
		\item Use os resultados das partes (a) e (b) para mostrar que 
		$$\dfrac{\partial z}{\partial x} = \dfrac{\partial z}{\partial r} \cos \theta - \displaystyle\frac{1}{r}\dfrac{\partial z}{\partial \theta} \sin \theta$$
		$$\dfrac{\partial z}{\partial y} = \dfrac{\partial z}{\partial r} \sin \theta + \displaystyle\frac{1}{r}\dfrac{\partial z}{\partial \theta} \cos \theta$$
		
		\item Use o resultado da parte (c) para mostrar que 
		
		 $$\left(\dfrac{\partial z}{\partial x}\right)^2 + \left(\dfrac{\partial z}{\partial y}\right)^2 = \left(\dfrac{\partial z}{\partial r}\right)^2 + \displaystyle\frac{1}{r^2}\left(\dfrac{\partial z}{\partial \theta}\right)^2$$
		
		\item Use o resultado da parte (c) para mostrar que se $z = f(x,y)$ satisfaz a equação de Laplace
		$$\dfrac{\partial^2 z}{\partial x^2} + \dfrac{\partial^2 z}{\partial y^2} = 0$$
		então $z = g(r, \theta)$ satisfaz a equação 
		$$\dfrac{\partial^2 z}{\partial r^2} + \displaystyle\frac{1}{r^2}\dfrac{\partial^2 z}{\partial \theta^2} + \displaystyle\frac{1}{r}\dfrac{\partial z}{\partial r} = 0$$
		
		e reciprocamente. A última equação acima é chamada de forma polar da equação de Laplace.
	\end{enumerate}		 
	 
	\end{enumerate}
	

\end{document}