\documentclass[11pt,a4paper]{article}

\usepackage{epsfig}
\usepackage{multicol}

\usepackage[utf8]{inputenc}
\usepackage[brazil]{babel}
\usepackage{fancyheadings}
\usepackage{amsmath}
\usepackage{enumerate}
\DeclareGraphicsExtensions{.png,.pdf}
\usepackage{amsmath, amsfonts, amssymb}
\usepackage{esint}
\usepackage{graphicx}
\usepackage{multicol}
\usepackage{tasks}
\usepackage[utf8]{inputenc}
\usepackage{mathrsfs} % Transformada de Laplace
\usepackage{indentfirst}

% As margens
\setlength{\textheight}{24.0cm}
\setlength{\textwidth}{17.5cm}
\setlength{\oddsidemargin}{2.0cm} % Margens reais desejadas
\setlength{\evensidemargin}{2.0cm} % 2+17.5+1.5=21cm (largura A4)
\setlength{\topmargin}{1.5cm} % 1.5+1.6+1.0+24.0+1.6=29.7cm
\setlength{\headheight}{1.6cm} % (altura A4)
\setlength{\headsep}{1.0cm}
\setlength{\columnsep}{1.5cm} % Coluna = 8cm ((17.5-1.5)/2)
\addtolength{\oddsidemargin}{-1in}
\addtolength{\evensidemargin}{-1in}
\addtolength{\topmargin}{-1in}
\setlength{\footskip}{0.0cm}


% Novos comandos
\newcommand{\limite}{\displaystyle\lim}
\newcommand{\integral}{\displaystyle\int}
\newcommand{\somatorio}{\displaystyle\sum}


\pagestyle{fancy}


\usepackage{lipsum}

\lhead{
\includegraphics[width=1cm]{brasao.png}
}

\rhead{ 
\sc\textbf{U}niversidade \textbf{F}ederal do \textbf{C}eará\\
Campus Quixadá\\ Monitoria de Cálculo II, III e EDO}

\cfoot{}

%\item $\somatorio_{k=0}^{n} t^k $, $0 < |t| < 1$
%$e^{i\theta} = cos\theta + i \, sin\theta$

\begin{document}

	\begin{center}
		\Large Lista 8 - Séries de Taylor e McLaurin
	\end{center}

	\begin{enumerate}
	
		\item Encontre a série de Maclaurin das funções abaixo:
		\begin{enumerate}
			\item $f(x) = e^x$
			\item $f(x) = e^{-x}$
			\item $f(x) = \sin x$
			\item $f(x) = \cos x$
			\item $f(x) = \cos (-x)$
		\end{enumerate}
	
	\item Usando Séries de Maclaurin determine a equação de Euler:
		$$e^{i\theta} = cos\theta + i \, sin\theta$$
		
		E conclua que $e^{i \pi} + 1 = 0$.
	
	\item Encontre a série de Taylor de $f(x) = e^x$ em $a = 2$.	
	
	\item Represente $f(x) = \sin x$ como a soma de sua série de Taylor centrada em $\displaystyle\frac{\pi}{3}$.
	
	\item Encontre a série de Maclaurin de $f(x) = (1 + x)^k$. onde $k$ é um número real qualquer.
	
	\item Determine se a série $\somatorio_{n=1}^{\infty} \displaystyle\frac{5}{2n^2 + 4n + 3} $ converge ou diverge.
	
	\item Teste a série $\somatorio_{n=1}^{\infty} \displaystyle\frac{\ln k}{k}$ quanto à convergência ou divergência.	
	
	\item Teste a série $\somatorio_{n=1}^{\infty} \displaystyle\frac{1}{2^n - 1}$ quanto à convergência ou divergência.
	
	\item Teste a série $\somatorio_{n=1}^{\infty} \displaystyle\frac{2n^2 + 3n}{\sqrt{5 + n^5}}$ quanto à convergência ou divergência.
	
	\item Teste a série $\somatorio_{n=1}^{\infty} (-1)^{n + 1} \displaystyle\frac{n^2}{n^3 + 1}$ quanto à convergência ou divergência.	
	
	\item Use a soma dos 100 primeiros termos para aproximar a soma da série abaixo e estime o erro envolvido nessa aproximação.
	$$\somatorio_{n=0}^{\infty} \displaystyle\frac{1}{n^3 + 1}$$ 
	
	\item Encontre a soma da série $\somatorio_{n=0}^{\infty} \displaystyle\frac{(-1)^n}{n!}$ com precisão de três casas decimais. 
	
	
	\end{enumerate}
	
%$\integral_0^{\infty} \left(\frac{x}{x^2 + 1} - \frac{C}{3x + 1}\right) \, dx$
\end{document}